\documentclass[a4paper,12pt]{article}% use option titlepage to get the title on a page of its own.

\usepackage[slovene]{babel}
\usepackage[utf8]{inputenc}
\usepackage[T1]{fontenc}
\usepackage{lmodern}
\usepackage{amsmath}
\usepackage{amssymb}
\usepackage[shortlabels]{enumitem}


\title{Analiza različnih modelov trgovanja na decentraliziranih borzah \\ \large Kratko poročilo za projekt pri predmetu Matematika z računalnikom}
\date{April 2024}
\author{Špela Bernardič in Nika Pavlič}

\begin{document}
\maketitle
%\section*{Plan dela}

Projekt delava pod mentorstvom Filipa Koprivca in Mateja Janežiča iz podjetja AFLabs, ki je projekt predlagal.

Projekt zahteva znanje programskega jezika solidity. Ker nama je programski jezik neznan sva vzpostavili dva repozitorija.
Prvi repozitorij je namenjen učenju - tu sva se oz. se še spoznavava s programskim jezikom solidity ter okoljem hardhat (za razvoj pametnih pogodb). Poglobiti sva se morali tudi teoretično in sicer naučiti sva se morali o samih osnovah blockchaina, varnosti, decentraliziranih financah, ERC\-20 žetonih ter nalaganju žetonov na omrežje.


Drugi repozitorij je namenjen dejanskemu projektu, kjer se bova posvetili decentraliziranim financam - specifično Uniswapu.

Uniswap je decentraliziran avtomatiziran likvidnostni protokol, zgrajen na Ethereumu. V protokol so vgrajene pogodbe, ki uporabnikom omogočajo trgovanje brez posrednikov. Protokol temelji na ponudnikih likvidnosti, ki ustvarjajo likvidnostne sklade, ti pa nato zagotavljajo likvidnost na celotni platformi, kar uporabnikom omogoča nemoteno izmenjavo med vsemi ERC-20 žetoni. 

Fokus projekta bosta Uniswap-v2 ter Uniswap-v3. Pripravili bova lastno kopijo Uniswap-v2, pripravili žetone in skripte, ki simulirajo izmenjavo. Analizirali bova Impermanent loss, ki nastane pri izmenjavi žetonov. Potem bova protokol Uniswap-v3 teoretično analizirali in ga uporabili - tudi zanj bova pripravili lastno kopijo, žetone in skripte, ki simulirajo izmenjavo. 

Cilj projekta je predvsem prepoznati razlike med Uniswap-v2 in Uniswap-v3 ter prikazati zakaj je slednji boljši. 

\end{document}


