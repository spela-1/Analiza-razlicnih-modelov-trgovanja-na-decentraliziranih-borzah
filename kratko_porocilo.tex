\documentclass[a4paper,12pt]{article}% use option titlepage to get the title on a page of its own.

\usepackage[slovene]{babel}
\usepackage[utf8]{inputenc}
\usepackage[T1]{fontenc}
\usepackage{lmodern}
\usepackage{amsmath}
\usepackage{amssymb}
\usepackage[shortlabels]{enumitem}


\title{Analiza različnih modelov trgovanja na decentraliziranih borzah \\ \large Kratko poročilo za projekt pri predmetu Matematika z računalnikom}
\date{April 2024}
\author{Špela Bernardič in Nika Pavlič}

\begin{document}
\maketitle
%\section*{Plan dela}

Projekt delava pod mentorstvom Filipa Koprivca in Mateja Janežiča iz podjetja AFLabs. \\

Pripravili sva dva repozitorija. Prvi je namenjen učenju, saj projekt zahteva znanje programskega jezika solidity ter okolja hardhat, s katerima se srečujeva prvič. Velik del projekta torej zavzema učenje za naju novega programskega jezika in okolja za razvoj pametnih pogodb. Poleg tega sva se morali poglobiti tudi v teoretično ozadje decentraliziranih financ, za kar sva uporabili gradivo, ki sta ga predlagala mentorja. Bolj podrobno: osnove blockchaina, varnosti, ERC\-20 žetonov, nalaganja pogodb na omrežje in uniswapa.\\

Drugi repozitorij je namenjen dejanskemu projektu, kjer se bova posvetili specifično Uniswapu. To je decentraliziran avtomatiziran likvidnostni protokol, zgrajen na Ethereumu. V protokol so vgrajene pogodbe, ki uporabnikom omogočajo trgovanje brez posrednikov. Protokol temelji na ponudnikih likvidnosti, ki ustvarjajo likvidnostne sklade, ti pa nato zagotavljajo likvidnost na celotni platformi, kar uporabnikom omogoča nemoteno izmenjavo med vsemi ERC-20 žetoni. \\

Fokus projekta bosta Uniswap-v2 ter Uniswap-v3. Pripravili bova lastno kopijo Uniswap-v2,  žetone in skripte, ki simulirajo izmenjavo. Analizirali bova Impermanent loss, ki nastane pri izmenjavi žetonov. Potem bova protokol Uniswap-v3 teoretično analizirali in ga uporabili - tudi zanj bova pripravili lastno kopijo, žetone in skripte, ki simulirajo izmenjavo. \\

Cilj projekta je priraviti poročilo in predstavitev, ki bosta zajemala razlike med Uniswap-v2 in Uniswap-v3 ter prikazati zakaj je slednji boljši. 

\end{document}


